\chapter{Verifikation}

\section{Zeitliche Abläufe}
Vorgeben: Menge A von Aktionen
\begin{description}
\item {Ereignis (hier)} Paar bestehend aus Aktion und Zeitpunkt\\aktion(e), zeit(e) für Ereignis e.
\end{description}
Beispiel: Schlacht bei Isis 333 v. Chr. $\rightarrow$ Aktion, Zeitpunkt
\subsubsection*{Idealisierende Annahmen:}
\begin{enumerate}
\item Alles findet praktisch am selben Ort statt, keine Probleme mit der Lichtgeschwindigkeit (30cm in 1ns).
	\begin{description}
	\item[Zeit (hier)] Newtonsche Zeit, Sie verläuft
		\begin{itemize}
		\item absolut d.h. unabhängig von Beobachter (sonst: spezielle Relativitätstheorie)
		\item stetig, d.h. ohne Sprünge (sonst Quantenmechanik)
		\item unbeeinflusst von der Umgebung (sonst: allg. Relativitätstheorie)
		\item Zeitpunkt = reale Zahl
		\end{itemize}
	\end{description}
\item Ein Ereignis hat die Dauer Null. Einen Zeitraum kann man darstellen durch die Ereignisse “Ende des Zeitraums“.
\item Gleichzeitige Ereignisse sind ausgeschlossen, d.h. zwei Ereignisse, die die zur gleichen Zeit stattfinden, sind gleich\\
$zeit(e)1 = zeit(e_2) \leftrightarrow e_1 = e2 $
\end{enumerate}

\begin{description}
\item[diskreter zeitlicher Ablauf (auch Geschichte)] Menge E von Ereignissen, so dass:
	\begin{enumerate}
		\item die Menge der Zeitpunkte E keinen Häufungspunkt hat
		\item die Menge der Zeitpunkte von E ein kleinstes Element hat
	\end{enumerate}
\end{description}
Interessant sind hier nicht die Zeitpunkte selber, sondern nur deren Lage zueinander, d.h. die Reihenfolge der Aktionen. (Wenn dies nicht der Fall ist und Termine eingehalten werden müssen $\rightarrow$ Echtzeitsystem)

\begin{description}
\item[Def: (Leslie Lamport 1978)] Ereignis $e_1$ kommt vor Ereignis $e_2$ :\\
$e_1 \rightarrow e_2 \leftrightarrow  zeit(e_1) < zeit(e_2)$
\end{description}

Beispiel: Hochmut  $\rightarrow$ Fall.\\
Es gilt:  $\rightarrow$ ist reflexiv, transitiv, total, fundiert (d.h. Wohlordnung)

Eine Relation $R \in E \times E$ auf der Menge $E$ heißt
\begin{itemize}
	\item irreflexiv, falls $\forall e \in E $ gilt $ (e, e) \notin R$
	\item transitiv, falls $\forall e_1, e_2, e_3 \in E $ gilt: Falls $(e_1,e_2) \in R $ und $ (e_2,e_3) \in R$, dann $(e_1,e_3) \in R$.
	\item total, falls $\forall e_1,e_2 \in E$ gilt: Falls $e_1 \neq e_2$, dann $(e_1, e_2) \in R$ oder $(e_2, e_1) \in R$
	\item fundiert, falls es keine unendliche Folge $(e_i)_{i \in \mathbb{N}}$ gibt mit $e_i \in E$ für alle $i \in \mathbb{N}$ und $(e_i, e_{i+1}) \in R$ für alle $i \in \mathbb{N}$
\end{itemize}

Einschub: R azyklisch, falls es keine endliche Folge $(e_1,..,e_n)$ gibt mit $(e_1,e_2) \in R, (e_2,e_3) \in R,...,(e_{n-1},e_n) \in R, (e_n,e_1) \in R$.
Falls R irreflexiv und transitiv ist, dann ist R auch azyklisch.

Für einen nicht leere Geschichte E sei min E definiert als das kleinste Element von E bezüglich $\rightarrow$, d.h. dasjenige $e \in E$ für das gilt:
\begin{equation*}
\forall f \in E\setminus{e}: e \rightarrow f
\end{equation*}

Tipp: Relation als Graph vorstellen mit Wegen.
\begin{itemize}
	\item Es existiert kein Weg der Länge 1 zu sich selber.
	\item Wenn es einen Weg von 1 zu 2 und 2 zu 3 gibt, dann existiert eine Abkürzung von 1 zu 3.
	\item Es gibt immer Weg von jedem zu jedem Knoten.
	\item Es existiert kein unendlicher Weg.
\end{itemize}

\begin{description}
\item[Implizite Definition] Definition durch eine charakterisierende Eigenschaft
\item[Wohldefiniertheit der implizierten Definition] Es gibt genau ein Objekt, dass die charakterisierende Eigenschaft erfüllt.
 (Bsp.: "Wurzel von x ist das, was quadriert x ergibt" ist nicht eindeutig (gar keine Lsg. bzw. mehrere))
\end{description}

Wohldefiniertheit von min E gilt, weil R total und E (mind.) ein kleinstes Element hat. (Ganze Zahlen sind z.B. total auf $<$, haben aber kein kleinstes Element)

Das i-te Element aus E ($E^i$) ist dann für $i \in \mathbb{N}, i \leq |E|$:
\begin{equation*}
E^i := \begin{cases}
\text{min E}&\text{, falls }i = 1\\
(E \setminus{\text{min E}})^{i-1}& \text{, sonst}
\end{cases}
\end{equation*}

Auch hier ist Wohldefiniertheit zu zeigen.\\
Projektion auf eine Menge B von Aktionen (“Sicht“):
\begin{equation*}
\pi_B(E) := e \in E | aktion(e) \in B
\end{equation*}
Zustand zum Zeitpunkt $t \in R$:
\begin{equation*}
z_t(E) := e \in E | zeit(e) \leq t
\end{equation*}


$\rightarrow$ für Zeiträume: Ende von Zeitraum A kommt vor Anfang von Zeitraum B: A $\rightarrow$ B Es gilt: Für Zeiträume ist $\rightarrow$ nicht total!
A $\rightarrow$ B v B $\rightarrow$ A $\leftrightarrow$ A und B überlappen nicht. (Wenn sich A und B überlappen gilt weder A  $\rightarrow$ B noch B  $\rightarrow$ A)

\begin{description}
	\item[Prozessalphabet] Menge der Aktionen, die der Thread p "sieht"
	\item[Gemeinsame Aktionen von $p_1$ und $p_2$] $\alpha(p_1) \cap \alpha(p_2)$
	\item[Einigkeit (engl. match)] Ereignisse mit gemeinsamen Aktionen finden gemeinsam statt:
		\begin{itemize}
			\item (1) $\pi_{\alpha(p_1) \cap \alpha(p_2)}(E_1 \cup E_2 = E_1 \cap E_2$ Gleichwertig zu (1) sind:\\
			\item (2) $\pi_{\alpha(p_1) \cap \alpha(p_2)}(E_1 \oplus E_2 = \emptyset$ // symmetr. Differenz: Vereinigung ohne Schnitt
			\item (3) $\pi_{\alpha(p_1)} = \pi_{\alpha(p_2)}$
		\end{itemize}
	\item[$E_i$ Ereignis von Thread i] Es gilt: $\forall e \in E_i: aktion(e) \in \alpha(p_i)$
	\item[Faire Mischung]$E_1 \cup E_2$
	\item[Gemeinsame Mischung] $\pi_{\alpha(p_i)}(E_1 \cap E_2) = E_i, für i \in {1,2}$, falls sich $p_1$ und $p_2$ einig sind. Es gilt: $E_1 \cup E_2$.
\end{description}


\section{Serielle Abläufe}
Wenn man nicht an den Zeitpunkten der Ereignisse interessiert ist, sondern nur an ihrer Lage zueinander, kann man statt einer Ereignismenge auch eine Aktionenfolge als Beschreibungsmittel für einen Ablauf nehmen.

Beispiele: Sei A = {a,b}. Endliche Folge (a, b, a) kann auch dargestellt werden als Funktion $f: {1,2,3} \rightarrow A$ mit $f(x) = \begin{cases} a &\text{, falls x = 1 oder x=3}\\ b &\text{sonst}\end{cases}$\\
Wertetabelle von f:\\
x     1   2   3\\
f(x)  a   b   a\\

Unendliche Folge (a, b, b, a, b, b, ...) als Funktion $f: \mathbb{N} \rightarrow A$ mit $f(x) = \begin{cases} a &\text{, falls x mod 3 = 1,} \\b &\text{sonst}\end{cases}$
$A^k$ k-Tupel von Elementen aus A und \\
${i \in \mathbb{N} | i \leq k} \rightarrow$ A Folgen der Länge k werden miteinander identifiziert.

\section{Faire Mischung}

\section{Sicherheits- und Liveness-Eigenschaften}

\section{Modellierung}