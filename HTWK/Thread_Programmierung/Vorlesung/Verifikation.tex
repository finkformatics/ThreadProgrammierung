\chapter{Verifikation}

\section{Zeitliche Abläufe}
Vorgeben: Menge A von Aktionen
\begin{description}
\item {Ereignis (hier)} Paar bestehend aus Aktion und Zeitpunkt\\aktion(e), zeit(e) für Ereignis e.
\end{description}
Beispiel: Schlacht bei Isis 333 v. Chr. $\rightarrow$ Aktion, Zeitpunkt
\subsubsection*{Idealisierende Annahmen:}
\begin{enumerate}
\item Alles findet praktisch am selben Ort statt, keine Probleme mit der Lichtgeschwindigkeit (30cm in 1ns).
	\begin{description}
	\item[Zeit (hier)] Newtonsche Zeit, Sie verläuft
		\begin{itemize}
		\item absolut d.h. unabhängig von Beobachter (sonst: spezielle Relativitätstheorie)
		\item stetig, d.h. ohne Sprünge (sonst Quantenmechanik)
		\item unbeeinflusst von der Umgebung (sonst: allg. Relativitätstheorie)
		\item Zeitpunkt = reale Zahl
		\end{itemize}
	\end{description}
\item Ein Ereignis hat die Dauer Null. Einen Zeitraum kann man darstellen durch die Ereignisse “Ende des Zeitraums“.
\item Gleichzeitige Ereignisse sind ausgeschlossen, d.h. zwei Ereignisse, die die zur gleichen Zeit stattfinden, sind gleich\\
$zeit(e)1 = zeit(e2) \leftrightarrow e1 = e2 $
\end{enumerate}

\section{Serielle Abläufe}

\section{Faire Mischung}

\section{Sicherheits- und Liveness-Eigenschaften}

\section{Modellierung}