\documentclass[a4paper]{scrreprt}

\usepackage[T1]{fontenc}
\usepackage[utf8]{inputenc}
\usepackage[ngerman]{babel}
\usepackage{amsthm}

\newtheorem{definition}{Definition}

\setlength\parindent{0pt}

\begin{document}

\tableofcontents

\chapter*{Gliederung}

\section*{Grundbegriffe}
Thread, Nicht-Determinismus, kritische Bereiche, Sperren

\section*{Verifikation}
Zeitliche Abläufe, serielle Abläufe, faire Mischung, Sicherheits- und Liveness-Eigenschaften, Modellierung

\section*{Synchronisation}
Signale, Beispiel: Erzeuger/Verbraucher (1), Semaphore, Beispiel: Erzeuger/Verbraucher (2), bedingte kritische Bereiche, Beispiel: Erzeuger/Verbraucher (3), wiederbetretbare Sperren, Leser/Schreiber-Problem

\section*{Feinkörnige Nebenläufigkeit}
Methoden, Beispiel: Mengen, grobkörnig, Beispiel: Mengen, feinkörnig, Beispiel: Mengen, optimistisch, Beispiel: Mengen, faul

\section*{Implementierung}
Atomare Befehle, Konsenszahlen, Zwischenspeicher, Bäckerei-Algorithmus

\section*{Transactional Memory}
Probleme mit Sperren, Transaktionen, Software Transactional Memory (STM): Transaktionsstatus, Transactional Thread, 2 Implementierungen

\chapter*{Literatur}
Maurice Herlihy, Nir-Shavit: The Art of Multiprocessor Programming (Morgan Kaufmann, 2008)\\
\\
Kalvin Lin, Larry Snyder: Principles of Parallel Programming (Addison Wesley)\\
\\
Greg Andrews: Concurrent Programming (Addison Wesley, 1991)\\
\\
Brian Goetz, u.a.: Java Concurrency in Practice (Addison Wesley)

\chapter{Grundbegriffe}

\section{Threads}

\begin{definition}[Prozess]
Sequentieller Rechenvorgang
\end{definition}

\begin{definition}[sequentiell]
Alle Rechenschritte laufen nacheinander in einer vorgegebenen Reihenfolge ab.
\end{definition}

\begin{definition}[Thread]
"`leichte"' Variante eines Prozesses
\end{definition}

Allgemeine Tendenz:
\begin{enumerate}
\item Systemkern möglichst "`schlank"' halten
\item Systemkern möglichst selten betreten
\end{enumerate}

Unterschied zu Prozess:
\begin{itemize}
\item Kein eigener Speicherbereich
\item Üblicherweise nicht vom Systemkern verwaltet ("`leight-weight process"'), vom Systemkern verwaltet
\end{itemize}

Vorteile:
\begin{itemize}
\item Wechsel zwischen Threads weniger aufwändig als Wechsel zwischen Prozessen
\item Threads benötigen weniger Speicher
\item Man kann viel mehr Threads ($\approx$ 10.000) als Prozesse ($\approx$ 100) laufen lassen.
\end{itemize}

Nachteil:\\
Anwendungsprogrammierer muss sich um Verwaltung der Threads kümmern.\\
\\
Viele Programmiersprachen bieten heutzutage Programmbibliotheken für Threads an (Beispiel: \emph{PThread} in C). Wir verwenden in dieser Veranstaltung \emph{Java} als Programmiersprache.

\begin{definition}[parallel]
Mehrere Threads laufen gleichzeitig auf verschiedenen Rechnerkernen.
\end{definition}

\begin{definition}[verschränkt (engl. interleaved)]
Threads laufen abwechselnd je ein Stück weit.
\end{definition}

\begin{definition}[nebeneinander laufend (auch: nebenläufig, engl. concurrent)]
Mehrere Threads laufen parallel oder miteinander verschränkt.
\end{definition}

Auch Mischformen sind möglich.\\
\\
Unterschied:\\
\begin{definition}[Rechenzeit (cpu time)]
Zeit, die der Prozessor mit Rechnen zubringt.
\end{definition}

\begin{definition}[Bearbeitungszeit (wall clock time)]
Umfasst auch Wartezeiten
\end{definition}

\subsubsection*{Amdahlsches Gesetz (Gene Amdahl, 1967):}
Wenn eine Aufgabe die Bearbeitungszeit $ a $ benötigt und der Anteil $ 0 \leq p \leq 1 $ davon parallelisierbar ist, dann benötigt sie auf $ n $ Prozessoren die Bearbeitungszeit

\begin{equation}
a \left( 1 - p + \frac{p}{n} \right).
\end{equation}

Beispiel:\\
$ p = \frac{9}{10} $, $ n = 100 $\\
\\
Beschleunigung (speed up): 
\begin{equation*}
 \frac{a}{a \left( 1 - p + \frac{p}{n} \right)} = \frac{1}{1 - \frac{9}{10} + \frac{9}{1000}} \approx 9,17
\end{equation*}

\begin{equation*}
\text{Sogar} \lim\limits_{n \to \infty} \frac{1}{1 - p + \frac{p}{n}} = \frac{1}{1 - p} = 10
\end{equation*}

Fazit: Der nicht-parallelisierbare Anteil dominiert die Bearbeitungszeit.

\section{Nicht-Determinismus}
\begin{definition}[Nicht-Determinismus]
Das Verhalten eines Systems hat Freiheitsgrade.
\end{definition}

Nicht-Determinismus hat zwei Anwendungen:
\begin{enumerate}
\item Möglichkeiten des Verhaltens der Systemumgebung zusammenfassen (engl. don't know nondeterminism)
\item Spielraum für Implementierungen vorsehen (engl. don't care nondeterminism)
\end{enumerate}

Hier: System von Threads\\
Man muss davon ausgehen, dass die Rechenschritte der Threads beliebig miteinander verschränkt sind. Die Reihenfolge der Schritte eines Threads ist durch sein Programm vorgegeben ("`Programm-Reihenfolge"').\\
Der Zeitplaner (engl. scheduler) legt zur Laufzeit fest, in welcher Reihenfolge die Schritte zweier Threads zueinander ablaufen. Man möchte den Zeitplaner in seiner Entscheidungsfreiheit nicht unnötig einschränken, sondern einen möglichst großen Spielraum lassen.\\
Man verlangt deshalb, dass das System von Threads korrekt zusammenarbeitet unabhängig davon, wie der Zeitplaner die Verschränkung bildet. Don't know nondeterminism aus der Sicht des Anwendungsprogrammierers, don't care nondeterminism aus der Sicht des Zeitplaners.\\
\\
Beispiel:\\
Thread 1 führt aus: \circlearound{1} \circlearound{2} \circlearound{3}\\
Thread 2 führt aus: \circlearound{a} \circlearound{b} \circlearound{c}\\
\\
Beispiele für mögliche Abläufe:
\begin{itemize}
\item \circlearound{1} \circlearound{a} \circlearound{2} \circlearound{b} \circlearound{3} \circlearound{c}
\item \circlearound{a} \circlearound{b} \circlearound{c} \circlearound{1} \circlearound{2} \circlearound{3}
\item \circlearound{a} \circlearound{1} \circlearound{2} \circlearound{3} \circlearound{b} \circlearound{c}
\item . . .
\end{itemize}

\includegraphics[width=.4\textwidth]{Nondeterminism}

Da bei jedem Test der Zeitplaner eine andere Ausführungsreihenfolge (Umstände des Wettrennens, engl. race conditions) wählen kann, ist der Test praktisch nicht reproduzierbar. Wegen der großen Anzahl möglicher Abläufe ist ein systematisches Testen aussichtslos ("`Zustandsexplosion"').
\chapter{Verifikation}

\section{Zeitliche Abläufe}
Vorgeben: Menge A von Aktionen
\begin{description}
\item {Ereignis (hier)} Paar bestehend aus Aktion und Zeitpunkt\\aktion(e), zeit(e) für Ereignis e.
\end{description}
Beispiel: Schlacht bei Isis 333 v. Chr. $\rightarrow$ Aktion, Zeitpunkt
\subsubsection*{Idealisierende Annahmen:}
\begin{enumerate}
\item Alles findet praktisch am selben Ort statt, keine Probleme mit der Lichtgeschwindigkeit (30cm in 1ns).
	\begin{description}
	\item[Zeit (hier)] Newtonsche Zeit, Sie verläuft
		\begin{itemize}
		\item absolut d.h. unabhängig von Beobachter (sonst: spezielle Relativitätstheorie)
		\item stetig, d.h. ohne Sprünge (sonst Quantenmechanik)
		\item unbeeinflusst von der Umgebung (sonst: allg. Relativitätstheorie)
		\item Zeitpunkt = reale Zahl
		\end{itemize}
	\end{description}
\item Ein Ereignis hat die Dauer Null. Einen Zeitraum kann man darstellen durch die Ereignisse “Ende des Zeitraums“.
\item Gleichzeitige Ereignisse sind ausgeschlossen, d.h. zwei Ereignisse, die die zur gleichen Zeit stattfinden, sind gleich\\
$zeit(e)1 = zeit(e2) \leftrightarrow e1 = e2 $
\end{enumerate}

\section{Serielle Abläufe}

\section{Faire Mischung}

\section{Sicherheits- und Liveness-Eigenschaften}

\section{Modellierung}
\chapter{Synchronisation}

\section{Signale}
\begin{description}
	\item[Synchronisation (hier)] dafür sorgen, dass gewisse Abläufe ausgeschlossen sind. Auch: Koordination.
	\item[Signal (auch: Handshake, Meldung, engl. Notification)] Hinweis an einen anderen Thread, dass er weitermachen kann.
\end{description}
Analogie:
\begin{itemize}
	\item Startschuss beim Wettlauf
	\item Staffel beim Staffellauf
	\item Anschlusszug muss warten
	\item Becher vor Kaffeezulauf
\end{itemize}
Ein Signal kann durch eine Sperre implementiert werden:
\begin{itemize}
	\item signalisieren (auch: melden) = freigeben
	\item warten = belegen
\end{itemize}
Das Signal soll garantieren, dass eine gewisse Reihenfolge eingehalten wird.
\begin{lstlisting}
P1:
	S1;
	freigeben(l);
P2:
	belegen(l);
	S2;


           S1
P1 -----|------|--|-----------
P2 -----|----------|--|------|
           warten        S2
\end{lstlisting}
l muss freigegeben worden sein, bevor es wieder belegt werden kann, also findet $S_1$ vor $S_2$ statt. Durch die Verwendung von Signalen schränkt man die Menge der Abläufe ein. Nachteil: \emph{weniger Parallelität}.\\
\\
\textbf{Extremfall:} nur noch eine Reihenfolge möglich; der Ablauf wird seriell. Abgesehen vom Koordinationsaufwand zu einem seriellen Programm dann gleichwertig.

\section[Beispiel: Erzeuger/Verbraucher (1)]{Beispiel: Erzeuger/Verbraucher-Problem, 1. Version}
Erzeuger und Verbraucher sind Threads. Der Erzeuger erzeugt Datenblöcke. Der Verbraucher holt die Datenblöcke ab und verarbeitet sie. Die erzeugten aber noch nicht verbrauchten Datenblöcke werden in einem Puffer (:= Warteschlange) zwischengespeichert.

\subsection*{1. Version}
\begin{tabular}{l r}
Erzeuger & 1\\
Verbraucher & 1\\
Puffergröße & 1
\end{tabular}\\
\\
Thread \emph{erz}:
\begin{lstlisting}
Wiederhole:
	herstellen(datenblock);
	einreihen(puffer, datenblock);
\end{lstlisting}
Thread \emph{verb}:
\begin{lstlisting}
Wiederhole:
	abholen(puffer, datenblock);
	verarbeiten(datenblock);
\end{lstlisting}

Prozeduren:
\begin{lstlisting}
	einreihen(puffer, datenblock):
1		belegen(leer);
2		kopieren(datenblock, puffer); // kopiert Datenblock in Puffer
3		freigeben(voll);

	abholen(puffer, datenblock):
4		belegen(voll);
5		kopieren(puffer, datenblock); // kopiert Puffer in Datenblock
6		freigeben(leer);
\end{lstlisting}

Hauptprogramm:
\begin{lstlisting}
	Sperre voll anlegen; // als belegt
	Sperre leer anlegen; // als belegt
	Threads erz und verb anlegen und laufen lassen;
0	freigeben(leer);
\end{lstlisting}

\subsubsection*{Kausalitätsgraph}
\begin{figure}[H]
	\begin{center}
		\includegraphics[width=.5\textwidth]{res/SynchronisationKausalitätsgraph}
		\label{pic:synkaus}
	\end{center}
\end{figure} 
$\longrightarrow$: Programmreihenfolge; $\dashrightarrow$: Reihenfolge erzwungen durch Signal.

\subsubsection*{Petri-Netz}
\begin{figure}[H]
	\begin{center}
		\includegraphics[width=\textwidth]{res/PetriNetz}
		\label{pic:synpetri}
	\end{center}
\end{figure} 
Bipartiter Graph: Zwei Sorten von Knoten; Pfeile nur zwischen verschiedenen Knotensorten.

\subsubsection*{Elementare Ereignis-Struktur}
Durch Abrollen des Kausalitätsgraphen:
\begin{figure}[H]
	\begin{center}
		\includegraphics[width=.7\textwidth]{res/ElementareEreignisstruktur}
		\label{pic:synelem}
	\end{center}
\end{figure} 

\subsubsection*{Signaldiagramme}
\begin{figure}[H]
	\begin{center}
		\includegraphics[width=\textwidth]{res/SignalDiagramme}
		\label{pic:synsignal}
	\end{center}
\end{figure} 
\begin{itemize}
	\item (2) und (5) werden als kritische Bereiche behandelt
	\item (2) und (5) werden nur abwechselnd ausgeführt
\end{itemize}
zu 1: Gegenseitiger Ausschluss gilt: $\neg$leer.frei $\lor$ voll.frei ist invariant.\\
zu 2: Folgt aus Programmierreihenfolge und 1.

\section{Semaphore}
\begin{description}
	\item[Semaphor] Datenstruktur mit Zustand l.frei $\in \mathbb{N}_0$ und Operationen "`belegen"' und "`freigeben"'.
	\item[belegen(l)] Wartet bis l.frei > 0 und setzt dann l.frei auf l.frei - 1.
	\item[freigeben(l)] Setzt l.frei auf l.frei + 1
\end{description}
Sperre ist Spezialfall mit l.frei $\in \{0, 1\}$.\\
Zweck: l.frei verschiedene Kopien eines Betriebsmittels werden verwaltet.\\
Zusammenhang zu Klammerausdrücken:\\
"`("' bedeutet "`freigeben(l)"'\\
"`)"' bedeutet "`belegen(l)"'\\
l.frei = Anzahl der noch offenen Klammern\\
\\
Beispiel:
\begin{lstlisting}
                  ( ( ) ( ) )
l.frei 2_|           _   _
       1_|         _| |_| |_
       0_|________|         |_______

\end{lstlisting}
\section[Beispiel: Erzeuger/Verbraucher (2)]{Beispiel: Erzeuger/Verbraucher-Problem, 2. Version}

\subsection*{2. Version}
\begin{tabular}{l r}
	Erzeuger & 1\\
	Verbraucher & 1\\
	Puffergröße & N (mit $N > 0$ beliebig)
\end{tabular}\\
\\
Threads erz und verb wie in Version 1.\\
\\
Prozeduren:
\begin{lstlisting}
	einreihen(puffer, datenblock):
1		belegen(nichtvoll); // I' gilt nun wieder
2		stock(puffer, datenblock); // Anfuegen des Datenblocks an Puffer hinten
3		freigeben(nichtleer); // I gilt
	
	abholen(puffer, datenblock):
4		belegen(nichtleer); // I' gilt
5		datenblock := top(puffer); // Liefert vordersten Datenblock des Puffers
6		pop(puffer);
7		freigeben(nichtvoll); // I gilt

	main():
		Leeren puffer anlegen;
		Semaphore nichtvoll und nichtleer erzeugen mit nichtvoll.frei = 0 und nichtleer.frei = 0;
		Threads erg und verb anlegen und laufen lassen;
		freigeben^N(nichtvoll); // nichtvoll wird N-mal freigegeben
		// I gilt
\end{lstlisting}

Invariante I: $ 0 \leq \text{ nichtvoll.frei } \leq N \land \text{ nichtleer.frei } + \text{ nichtvoll.frei } \leq N $\\
Invariante I': $ 0 \leq \text{ nichtvoll.frei } \leq N \land \text{ nichtleer.frei } + \text{ nichtvoll.frei } \leq N - 1 $\\
Invariante I'': $ 0 \leq \text{ nichtvoll.frei } \leq N \land \text{ nichtleer.frei } + \text{ nichtvoll.frei } \leq N - 2 $\\
I'' gilt, wenn beide Threads belegen, aber noch nicht freigeben aufgerufen haben.

\section{Bedingte kritische Bereiche}
Ein kritischer Bereich soll nur betreten werden, wenn eine gewisse Bedingung B an die gemeinsame Variable gilt. Wie implementiert man das?
\begin{enumerate}
	\item B vor dem Betreten des kritischen Bereichs überprüfen.\\Problem: B kann beim Betreten des kritischen Bereiches bereits wieder verletzt sein.
	\item B im kritischen Bereich überprüfen.\\Problem: Solange B nicht gilt, soll der Thread warten. Weil er sich im kritischen Bereich befindet, können andere Threads die gemeinsame Variable nicht ändern und damit den Wert von B.
\end{enumerate}
Mit kritischen Bereichen kann man das Problem nicht lösen. Abhilfe: neues Konstrukt.

\section[Beispiel: Erzeuger/Verbraucher (3)]{Beispiel: Erzeuger/Verbraucher-Problem, 3. Version}

\subsection*{3. Version}
\begin{tabular}{l r}
	Erzeuger & n\\
	Verbraucher & m\\
	Puffergröße & N (mit $N > 0$ beliebig)
\end{tabular}\\
\\
\begin{lstlisting}
einreihen(puffer, datenblock):
	kritisch l {
		warte auf laenge(puffer) < N;
		stock(puffer, datenblock);
	}

abholen(puffer, datenblock):
	kritisch l {
		warte auf laenge(puffer) > 0;
		datenblock := top(puffer);
		pop(puffer);
	}

// laenge(puffer) liefert die Anzahl der Datenbloecke im Puffer

main():
	Erzeugen des Puffers (leer);
	Erzeugen der Sperre l fuer den Puffer;
	Erzeugen und Starten der Threads;

// Version mit Bedingungsvariablen
einreihen(puffer, datenblock):
	belegen(l);
	solange laenge(puffer) < N wiederhole:
		wait(nichtvoll);
	stock(puffer, datenblock);
	signalAll(nichtleer);
	falls laenge(puffer) < N, dann:
		signalAll(nichtvoll);
	freigeben(l);

abholen(puffer, datenblock):
	belegen(l);
	solange laenge(puffer) > 0 wiederhole:
		wait(nichtleer);
	datenblock := top(puffer);
	pop(puffer);
	signalAll(nichtvoll);
	falls laenge(puffer) > 0, dann:
		signalAll(nichtleer);
	freigeben(l);
\end{lstlisting}

\section{Wiederbetretbare Sperren}
\begin{description}
	\item[Wiederbetretbare Sperre (engl. reentrant lock)] Ein Thread darf die Sperre mehrfach erwerben.
\end{description}
Zweck: Innerhalb eines kritischen Bereiches darf man eine Prozedur aufrufen, die wieder einen kritischen Bereich für dieselbe gemeinsame Variable enthält. $\Rightarrow$ bequemere Programmierung.\\
Die inneren kritischen Bereiche sollen dazu wirkungslos sein.\\
\\
Beispiel:
\begin{lstlisting}
belegen(l);
S1;
belegen(l);
S2;
freigeben(l);
S3;
freigeben(l);
// Von S1 bis S3 kritischer Bereich. Innere belegen(l) und freigeben(l) wirkungslos.
\end{lstlisting}
\textbf{Bemerkung:} Wiederbetretbare Sperren und Semaphore sind inkompatibel zueinander.

\section{Leser/Schreiber-Problem}
Mehrere Threads greifen lesend oder schreibend auf eine gemeinsame Variable zu (Courtois et al, 1971). Mehrere Threads können lesend auf die gemeinsame Variable zugreifen, ohne sich gegenseitig zu stören.

\begin{description}
	\item[Lese/Schreib-Konflikt] Eine gemeinsame Variable darf nicht gleichzeitig gelesen und geschrieben werden.
	\item[Schreib/Schreib-Konflikt] Eine gemeinsame Variable darf nicht von mehreren Threads gleichzeitig geschrieben werden.
\end{description}

\textbf{2 Varianten:}
\begin{enumerate}
	\item Ein Leser muss nur dann warten, wenn gerade ein Schreiber aktiv ist: Leser haben Vorrang.
	\item Ein Schreiber muss nur auf Leser und Schreiber warten, die gerade aktiv sind: Schreiber haben Vorrang.
\end{enumerate}
Wenn (1) und (2) abwechselnd verwendet werden, bekommt man Fairness.
\begin{center}
	\begin{tabular}{l r|c c}
		\ & \ & \multicolumn{2}{c}{\textbf{Leser}} \\ 
		\ & \ & 0 & $\geq 1$ \\ \hline
		\multirow{3}{*}{\textbf{Schreiber}} & 0 & i. O. & i. O. \\
		\ & 1 & i. O. & LS \\
		\ & $\geq 2$ & SS & LS + SS
	\end{tabular}
\end{center}
\chapter{Feinkörnige Nebenläufigkeit}
Problem: Bei einem hohen Grad an Nebenläufigkeit wird der Zugriff auf das gemeinsame Objekt zum Flaschenhals. Die Threads können nur nacheinander zugreifen.\\
Abhilfe: 4 Techniken.
\begin{enumerate}
	\item \textbf{Feinkörnige Synchronisation:} Statt das Objekt zu sperren, werden nur die betroffenen Komponenten gesperrt. Nur wenn zwei Threads auf dieselbe Komponente zugreifen wollen, muss einer warten.
	\item \textbf{Optimistische Synchronisation:} Während der Suche nach einer bestimmten Komponente des Objekts werden keine Sperren erworben. Sobald die Komponente gefunden wurde, diese Komponente sperren und überprüfen, ob sich der Kontext inzwischen verändert hat.
	\item \textbf{Faule Synchronisation:} Löschen einer Komponente wird in zwei Phasen durchgeführt:
	\begin{enumerate}
		\item Als gelöscht markieren, z.B. durch Setzen eines gewissen Bits ("`logisches Löschen"').
		\item Aus der Datenstruktur aushängen ("`physikalisches Löschen"').
	\end{enumerate}
	\item \textbf{Nicht-blockierende Synchronisation:} Statt Sperren werden atomare Operationen verwendet.
\end{enumerate}

\section[Mengen mit verketteten Listen]{Beispiel: Mengen implementiert durch verkettete Listen}
Mengen-Schnittstelle:
\begin{lstlisting}
Typ Set<T>
Methoden
    add(T x) // x zu Menge this hinzufuegen
    remove(T x) // x aus Menge this entfernen
    containes(T x) // Wahrheitswert von "this enthaelt x"
\end{lstlisting}
Implementierung von Mengen durch verkettete Listen mit Wächtern, sortiert nach Streuwert.\\
\\
Listenelement (Typ Node<T>) habe folgende Attribute:\\
\begin{description}
	\item[T item] das Element der Menge
	\item[int key] der Streuwert des Elements
	\item[Node<T> next] Zeiger auf das nächste Element
\end{description}

\begin{description}
	\item[Wächter (engl. Sentinel)] "`künstliches"' Listenelement, das Anfang oder Ende der Liste markiert.
\end{description}

Beispiel:\\ % Bild einfügen linkedlist (von Handy)
\\
Die Streuwerte sind sortiert: $ -\infty \leq 1 \leq 25 \leq +\infty $.\\
\\
Klasse List<T> mit Attribut head vom Typ Node<T>.\\
Invariante: Wächter werden weder hinzugefügt noch gelöscht. Die Listenelemente sind nach Streuwert sortiert.

\section{Implementierung mit Feinkörniger Synchronisation}
In Java:
\lstset{language=Java,tabsize=4}
\begin{lstlisting}
class List<T> {
	...
	public boolean add(T x) {
		int k = x.hashCode(); // Streuwert
		Node<T> pred, curr;
		try {
			pred = this.head;
			curr = pred.next;
			pred.lock(); // Assume implementation
			curr.lock();
			while (curr.key < k) {
				pred.unlock();
				pred = curr;
				curr = pred.next;
				curr.lock();
			}
			if (curr.key != k) {
				Node<T> node = new Node<>(x);
				node.next = curr;
				pred.next = node;
			}
		} finally {
			curr.unlock();
			pred.unlock();
		}
	}
}
\end{lstlisting}

Methoden remove und contains ähnlich.\\
\\
Eine Sperre genügt nicht.\\
Beispiel dazu:\\% Bild einfügen sperregenügtnicht (handy)
\\
Thread 1 will b löschen:
\begin{itemize}
	\item Sperre in a erwerben; a.next auf c setzen
\end{itemize}
Thread 2 will c löschen:
\begin{itemize}
	\item Sperre in b erwerben; b.next auf d setzen
\end{itemize}
Wirkung: Nur b wird gelöscht!\\
\\
Um b zu löschen, muss die Sperre in a und in b erworben werden, und entsprechend um c zu löschen, muss die Sperre in b und in c erworben werden. $\Rightarrow$ Konflikt!

\section{Implementierung mit optimistischer Synchronisation}

In Java:
\begin{lstlisting}
class OptList<T> {
	...
	public boolean add(T x) {
		int k = x.hashCode();
		Node<T> pred, curr;
		boolean done = false;
		while (!done) {
			pred = this.head;
			curr = pred.next;
			while (curr.key < k) {
				pred = curr;
				curr = pred.next;
			}
			try {
				pred.lock();
				curr.lock();
				if (this.validate(pred, curr)) {
					if (curr.key != k) {
						Node<T> node = new Node<>(x);
						node.next = curr;
						pred.next = node;
					}
					done = true;
				}
			} finally {
				curr.unlock();
				pred.unlock();
			}
		}
	}
	
	private boolean validate(Node<T> pred, Node<T> curr) {
		Node<T> node = this.head;
		while (node.key < pred.key) {
			node = node.next;
		}
		return node == pred && curr = node.next;
	}
}
\end{lstlisting}
Diskussion: Optimistisches Synchronisation lohnt sich, wenn zweimaliges Durchlaufen der Liste billiger ist als einmaliges Durchlaufen mit Setzen von Sperren. Belegen und Freigeben sind aufwendig.

\section{Beispiel: Mengen, faul}
\chapter{Implementierung}

\section{Atomare Maschinenbefehle}
Wie werden Sperren implementiert?

\subsubsection*{Naiver Versuch}
\begin{lstlisting}
belegen(l):
	Solange !l.frei gilt, wiederhole:    (1)
		warte einen Augenblick;          (2)
	setze l.frei = false;                (3)
\end{lstlisting}
Beispielablauf für 2 Threads, die versuchen, belegen(l) aufzurufen. Sei zu Beginn l.frei = true.
\begin{center}
\begin{tabular}{c c|c}
$P_1$ & $P_2$ & l.frei \\ \hline
(1) & \ & \ \\ 
\ & (1) & \ \\ 
(3) & \ & \ \\
\ & (3) & \ 
\end{tabular}
\end{center}
$\rightarrow$ Beide Threads sind im kritischen Bereich!\\
$\Rightarrow$ (1)(2)(3) muss selber wieder ein kritischer Bereich sein.\\
\\
Man benötigt einen speziellen Maschinenbefehl, z.B.:
\begin{lstlisting}
getAndSet(c, b, v):
	b := c;
	c := v;
\end{lstlisting}
Zwei Ausführungen dieses Maschinenbefehls müssen immer unter gegenseitigem Ausschlusds stattfinden, d.h. der Maschinenbefehl muss atomar (komplett und unteilbar) sein. Die Hardware muss dafür sorgen ("`Arbitrierung"').

\pagebreak

\subsubsection*{Implementierung mit getAndSet}
\begin{lstlisting}
belegen(l):
	boolean b;
	getAndSet(l.frei, b, false);
	solange !b wiederhole:
		warte einen Augenblick;
		getAndSet(l.frei, b, false);

freigeben(l):
	boolean b;
	getAndSet(l.frei, b, true);
\end{lstlisting}

Alternativen:

\begin{lstlisting}
getAndInc(c, b):
	b := c;
	c := c + 1;
getAndDec(c, b):
	b := c;
	c := c - 1;
compareAndSet(c, e, v, b): // b := Wahrheitswert
	Falls c = e, dann:
		c := v;
		b := true;
	Sonst:
		b := false;
\end{lstlisting}
(Befehl CMPXCHG (compare exchange) auf Intel Pentium)

\subsubsection*{Implementierung mit compareAndSet}

\begin{lstlisting}
belegen(l):
	boolean b;
	compareAndSet(l.frei, true, false, b);
	Solange !b wiederhole:
		warte einen Augenblick;
		compareAndSet(l.frei, true, false, b);

freigeben(l):
	boolean b;
	compareAndSet(l.frei, false, true, b);
\end{lstlisting}

\textbf{Volatile} (engl. für flüchtig), Schlüsselwort in Java.\\
Beispiel:
\begin{lstlisting}
volatile int x;
\end{lstlisting}
x ist damit als gemeinsame Variable gekennzeichnet. Übliche Optimierungen des Compilers für lokale Variablen sind ausgeschlossen. Lese- und Schreibzugriffe auf x sind zueinander atomar ("`atomares Register"'). Die Hardware sorgt für Atomarität.

\pagebreak

\section{Konsenszahlen}

Konsensproblem:
\begin{enumerate}
	\item Im Hauptprogramm: init(c); Gemeinsame Variable c wird initialisiert.
	\item Jeder Thread ruft höchstens ein Mal entscheide(c, v, a) auf. (c: gemeinsame Variable, v: Vorschlag vom Typ T, a: Variable vom Typ T)
	\item Der Aufruf entscheide(c, v, a) gibt an a einen Wert mit folgenden Eigenschaften:
	\begin{itemize}
		\item Einigkeit: Jeder Thread bekommt denselben Wert von a
		\item Gültigkeit: Der Wert in a wurde von mindestens einem Thread vorgeschlagen.
	\end{itemize}
\end{enumerate}

\begin{description}
	\item[n-Konsensproblem] Konsensproblem mit n beteiligten Threads
\end{description}

Es gilt:
\begin{itemize}
	\item Mit dem n-Konsensproblem löst man auch das k-Konsensproblem für jedes k < n.
	\item Das 1-Konsensproblem ist trivial löstbar: a := v implementiert entscheide(c, v, a).
\end{itemize}

Die Konsenszahl für eine Klasse (Sprache) K ist definiert als:
\begin{equation}
\begin{cases}
\infty & \text{falls K das n-Konsensproblem für alle $n \in \mathbb{N}$ löst}\\
n & \text{falls $n \in \mathbb{N}$ maximal, sodass K das n-Konsensproblem löst}
\end{cases}
\end{equation}

Satz: Herlihy, 1991:

\begin{center}
\begin{tabular}{l|c}
Klasse K & Konsenszahl K(K)\\ \hline
atomare Register & 1\\
Warteschlangen & 2\\
Common-2-Operationen & 2\\
compareAndSet & $\infty$
\end{tabular}
\end{center}

n-Konsens mit \emph{compareAndSet} und \emph{get}:\\
(Einfaches Konsensproblem: Jeder schlägt sich selber vor)\\
\begin{lstlisting}
init(c):
    Setze c = -1.

entscheide(c, i, a): // mit i: Thread-ID des Aufrufers
    boolean b;
    compareAndSet(c, -1, i, b);
    Falls b gilt, dann:
        a := i;
    Sonst:
        a := get(c); // Kann auch ohne Fallunterscheidung
                     // angewandt werden, da fuer b true gilt:
                     // i == get(c)
\end{lstlisting}

\subsubsection*{Read/Modify/Write-Operation:}
rmw(c, b, f): (mit c ist gemeinsame Variable mit Wert vom Typ T, b ist Ergebnisvariable mit Wert von Typ T und f ist Modifikationsfunktion f: T $\rightarrow$ T).\\
b := c;\\
c := f(c);\\

Es gilt: \\
getAndSet(c, b, v) = rmw(c, b, $\lambda$x . v)\\
getAndInc(c, b) = rmw(c, b, $\lambda$x . x + 1)\\

Schar F von Funktionen von T nach T heißt \emph{Common2}, falls:

\begin{align}
	f(g(x)) & = f(x) \text{ (f absorbiert g) oder}\\
	g(f(x)) & = g(x) \text{ oder}\\
	f(g(x) & = g(f(x))	
\end{align}
für alle $f, g \in F, x \in T$ (trivial für f = g).\\
\\
F heißt \emph{nicht-trivial}, falls $ F \neq \{id\} $ mit $ F $ ist nichtleer, d.h. $ F \backslash \{id\} \neq \emptyset $.\\
\\
Beispiel: $ F = \{\lambda x\ .\ x + 1, \lambda x\ .\ x - 1\} = \{s, p\} $.\\
Es gilt: $ s(p(x) = x = p(s(x)) $ für alle $ x \in \mathbb{Z} $. Also ist $ F $ Common2. Damit Konsenszahl $\leq$ 2. Da $ F $ nicht-trivial, ist Konsenszahl = 2. 

\section{Zwischenspeicher}

\begin{description}
	\item[Zwischenspeicher (ZSP, engl. cache)] schneller, kleiner Speicher auf dem Prozessorchip.
\end{description}
Bemerkung: Herkunft des Begriffs "`cache"': Versteck der Beute eines Einbrechers.\\
\\
Verwendung:\\
Nachdem der Prozessor das erste Mal auf eine gewisse Arbeitsspeicherzelle lesend zugegriffen hat, speichert er den Wert in seinem ZSP. Wenn er das nächste Mal lesend auf dieselbe Adresse zugreifen will, findet er das Ergebnis in seinem ZSP ("`Treffer"', engl. match). Er braucht dazu nicht auf den BUS zuzugreifen.

Um schreibend auf eine Arbeitsspeicherzelle zuzugreifen, speichert der Prozessor das Wort zunächst in seinen ZSP. Nur wenn ein anderer Prozessor auf dieselbe Speicherzelle lesend zugreifen will, muss das Wort in den Arbeitsspeicher geschrieben werden.

\subsubsection*{Vorteil des ZSP:}
Weniger Zugriffe auf den Arbeitsspeicher nötig, damit schneller und der BUS ist weniger belastet.

Der ZSP lohnt sich, wenn im Programm häufig dicht hintereinander Zugriffe auf dieselbe Adresse vorkommen ("`Lokalität"').

Um den Verwaltungsaufwand gering zu halten, ist der ZSP in sogenannte \emph{Speicherzeilen} (engl. cache lines) organisiert. Sobald der ZSP voll ist, wird es nötig, manche Zeilen auszuwerfen (engl. to evict) um Platz zu schaffen.

\begin{description}
	\item[Kohärenz] Jeder Lesezugriff auf den ZSP liefert den zuletzt geschriebenen Wert.
\end{description}

Kohärenz bedeutet praktisch, dass sich durch die Einführung des ZSP nichts am Verhalten des Systems ändert.

Um Kohärenz zu erreichen, verwendet man ein Kohärenz-Protokoll, z.B. das MESI-Protokoll.

\subsubsection*{MESI-Protkoll:}

Jede Speicherzeile hat einen Modus:
\begin{description}
	\item[Modified] Zeile wurde verändert. Kein anderer Prozessor hat diese Zeile in seinem ZSP.
	\item[Exclusive] Zeile ist unverändert. Kein anderer Prozessor hat diese Zeile in seinem ZSP.
	\item[Shared] Zeile ist unverändert. Andere Prozessoren können diese Zeile in ihrem ZSP haben.
	\item[Invalid] Zeile enthält eine verwertbaren Daten.
\end{description}
Beispiel-Ablauf:\\
A, B, C seien Prozessoren, M sei ein Arbeitsspeicherblock.
\begin{figure}[H]
	\begin{center}
		\includegraphics[width=.5\textwidth]{res/mesi_01}
		\caption{A liest Adresse von a.}
		\label{pic:mesi01}
	\end{center}
\end{figure} 
\begin{figure}[H]
	\begin{center}
		\includegraphics[width=.5\textwidth]{res/mesi_02}
		\caption{B liest Adresse von a; A antwortet.}
		\label{pic:mesi02}
	\end{center}
\end{figure} 
\begin{figure}[H]
	\begin{center}
		\includegraphics[width=.5\textwidth]{res/mesi_03}
		\caption{B schreibt auf Adresse a und informiert alle darüber.}
		\label{pic:mesi03}
	\end{center}
\end{figure} 
\begin{figure}[H]
	\begin{center}
		\includegraphics[width=.5\textwidth]{res/mesi_04}
		\caption{A liest von Adresse a, das führt zu Anfrage an alle, B sendet Daten.}
		\label{pic:mesi04}
	\end{center}
\end{figure} 

\begin{description}
	\item[False Sharing] gemeinsame Speicherzelle, obwohl sich die Daten darin nicht überlappen
\end{description}

Im ZSP von B:
\begin{figure}[H]
	\begin{center}
		\includegraphics[width=.5\textwidth]{res/false_sharing_01}
		\caption{Speicherzeile für Adresse a.}
		\label{pic:falsesharing}
	\end{center}
\end{figure} 
False Sharing führt unnötig häufig zu Modus I.\\
\\
Daten, die nebeneinander verwendet werden, sollten in verschiedenen Speicherzeilen liegen.\\
\\
Verhalten mit getAndSet:\\
getAndSet(c, b, true)\\
Dabei ausgeführte Aktionen:
\begin{enumerate}
	\item c lesen
	\item b schreiben
	\item c schreiben
\end{enumerate}
\begin{figure}[H]
	\begin{center}
		\includegraphics[width=.5\textwidth]{res/mesi_05}
		\caption{Zustand vorher.}
		\label{pic:mesi05}
	\end{center}
\end{figure} 
\begin{figure}[H]
	\begin{center}
		\includegraphics[width=.5\textwidth]{res/mesi_06}
		\caption{A führt (1) aus, Wert von c in ZSP von A ist bereits aktuell, keine Änderung.}
		\label{pic:mesi06}
	\end{center}
\end{figure} 
\begin{figure}[H]
	\begin{center}
		\includegraphics[width=.5\textwidth]{res/mesi_07}
		\caption{A führt (2) aus.}
		\label{pic:mesi07}
	\end{center}
\end{figure} 
\begin{figure}[H]
	\begin{center}
		\includegraphics[width=.5\textwidth]{res/mesi_08}
		\caption{A führt (3) aus.}
		\label{pic:mesi08}
	\end{center}
\end{figure} 
\begin{figure}[H]
	\begin{center}
		\includegraphics[width=.5\textwidth]{res/mesi_09}
		\caption{B führt (1) aus.}
		\label{pic:mesi09}
	\end{center}
\end{figure} 
\begin{figure}[H]
	\begin{center}
		\includegraphics[width=.5\textwidth]{res/mesi_10}
		\caption{B führt (2) aus.}
		\label{pic:mesi10}
	\end{center}
\end{figure} 
\begin{figure}[H]
	\begin{center}
		\includegraphics[width=.5\textwidth]{res/mesi_11}
		\caption{B führt (3) aus, keine Änderung, denn der Wert in c verändert sich nicht dabei.}
		\label{pic:mesi11}
	\end{center}
\end{figure} 

Alle getAndSet-Aufrufe der Warteschleife können ohne BUS-Zugriff abgearbeitet werden.

\section{Bäckerei-Algorithmus}
\begin{description}
	\item[Bäckerei-Algorithmus (engl bakery algorithm)] von Leslie Lamport 1974 publiziert; Implementierung von Sperren mit atomaren Registern.
\end{description}

Analogie: Jeder, der (in Amerika) eine Bäckerei betritt, zieht zuerst eine laufende Nummer. Der Kunde mit der niedrigsten Nummer wird als nächste bedient.
\pagebreak

Pseudocode mit einer Sperre:
\begin{lstlisting}[escapeinside={(*}{*)}]
Typ Thread ID = {0, ..., n - 1}; // n Threads
volatile flag: boolean[ThreadID]; // initialisiert mit false
volatile label: long[ThreadID]; // initialisiert mit 0

Prozedur belegen():
    int i := Nummer des aufrufenden Threads;
    flag[i] := true;
    label[i] := max(label[0], ..., label[n - 1]) + 1;
    Warte solange (* $\exists k \neq i : \text{flag}[k] \land \left(\text{label}[k], k\right) <_\text{lex} \left(\text{label}[i], i\right)$ *)
    
Prozedur freigeben():
    flag[Nummer des aufrufenden Threads] := false;
\end{lstlisting}
\textbf{Behauptung:} Der Bäckerei-Algorithmus hat die Fortschritt-Eigenschaft.\\
\textbf{Beweis:} Der Thread i mit dem kleinsten Paar (label[i], i) wartet nicht. Es gibt so ein i, denn $<_\text{lex}$ ist eine Wohlordnung. Damit hat jede nicht-leere Menge ein kleinstes Element.\\
\\
\textbf{Behauptung:} Der Bäckerei-Algorithmus ist FCFS (First Come First Serve).\\
\textbf{Beweis:} Falls Thread i den Torweg verlässt, bevor Thread j ihn betritt, dann gilt:
\begin{align*}
	& w_i\left(label[i], v\right) \rightarrow\\
	& r_j\left(label[i], v\right) \rightarrow\\
	& w_j\left(label[j], v'\right) \text{ mit } v < v'\\
	& r_j\left(flag[i], true\right)
\end{align*}
Dabei bedeutet $w_i\left(label[i], v\right)$: Schreibzugriff von Thread $i$ auf die Variable $label[i]$; der geschriebene Wert ist v.\\
Es gilt $flag[i] \land \left(flat[i], i\right) <_\text{lex} \left(label[j], j\right)$.\\
Aus Fortschritt und FCFS folgt Fairness.\\
\\
\textbf{Behauptung:} Der Bäckerei-Algorithmus erfüllt gegenseitigen Ausschluss.\\
\textbf{Beweis:} Durch Widerspruch (grundsätzliche Methode: Man behauptet, zwei Threads seien simulatan im kritischen Bereich. Herbeiführung von Widerspruch). Angenommen Threads i und j sind nebeneinander im kritischen Bereich. O.B.d.A. gilt: $\left(label[i], i\right) <_\text{lex} \left(label[j], j\right)$. Sobald Thread j die Warteschleife verlassen hat, gilt:
\begin{equation*}
	flag[i] = false \text{ (1)}
\end{equation*}
oder
\begin{equation*}
	\left(label[j], j\right) <_\text{lex} \left(label[i], i\right) \text{ (2)}
\end{equation*}
Die Werte von i und j sind fest. Der Wert von label[j] ändert sich nicht mehr bis zum Betreten des kritischen Bereichs. Der Wert von label[i] kann höchstens größer werden.\\
Wenn also (2) beim Verlassen der Warteschleife gilt, dann auch im kritischen Bereich. Widerspruch!\\
Also gilt (1). Deswegen 
\begin{align*}
&	r_j\left(\text{label}[i], \_\right) \rightarrow \text{ (\_: gelesener Wert ist irrelevant)}\\
&	w_j\left(\text{label}[j], v\right) \rightarrow\\
&	r_j\left(\text{flag}[i], false\right) \rightarrow\\
&	w_i\left(\text{flag}[i], true\right) \rightarrow\\
&	r_i\left(\text{label}[j], v\right) \rightarrow\\
&	w_i\left(\text{label}[i], v'\right)
\end{align*}
mit $v < v'$, also $\text{label}[j] < \text{label}[i]$. Widerspruch!\\
\\
Nachteil: Falls nur atomare Lese- und Schreib-Operationen zur Verfügung stehen ("`atomare Register"'), sind für n Threads Lese- und Schreibzugriffe auf mindestens n Speicherzellen notwendig (Burns/Lynch 1993).\\
Grund: Jeder Thread benötigt eine Speicherzelle, auf die nur er schreibt. Sonst kann ein anderer Thread das überschreiben, was ein anderer geschrieben hat.\\
Es muss mindestens n + 1 unterscheidbare Zustände geben: 
\begin{enumerate}
	\item kein Thread befindet sich im kritischen Bereich
	\item Thread i befindet sich im kritischen Bereich
\end{enumerate}
\chapter{Transactional Memory}

\section{Probleme mit Sperren}

\section{Transaktionen}

\section{Software Transactional Memory (STM)}

\subsection{Transaktionsstatus}

\subsection{Transactional Thread}

\subsection{Zwei Implementierungen}

\end{document}