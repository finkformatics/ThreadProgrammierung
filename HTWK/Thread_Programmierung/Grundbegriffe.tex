\chapter{Grundbegriffe}

\section{Threads}

\begin{definition}[Prozess]
Sequentieller Rechenvorgang
\end{definition}

\begin{definition}[sequentiell]
Alle Rechenschritte laufen nacheinander in einer vorgegebenen Reihenfolge ab.
\end{definition}

\begin{definition}[Thread]
"`leichte"' Variante eines Prozesses
\end{definition}

Allgemeine Tendenz:
\begin{enumerate}
\item Systemkern möglichst "`schlank"' halten
\item Systemkern möglichst selten betreten
\end{enumerate}

Unterschied zu Prozess:
\begin{itemize}
\item Kein eigener Speicherbereich
\item Üblicherweise nicht vom Systemkern verwaltet ("`leight-weight process"'), vom Systemkern verwaltet
\end{itemize}

Vorteile:
\begin{itemize}
\item Wechsel zwischen Threads weniger aufwändig als Wechsel zwischen Prozessen
\item Threads benötigen weniger Speicher
\item Man kann viel mehr Threads ($\approx$ 10.000) als Prozesse ($\approx$ 100) laufen lassen.
\end{itemize}

Nachteil:\\
Anwendungsprogrammierer muss sich um Verwaltung der Threads kümmern.