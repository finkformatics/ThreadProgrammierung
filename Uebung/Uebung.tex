\documentclass[a4paper]{scrartcl}

\usepackage[T1]{fontenc}
\usepackage[utf8]{inputenc}
\usepackage[ngerman]{babel}
\usepackage[hidelinks]{hyperref}
\usepackage{amsmath}
\usepackage{amssymb}
\usepackage{amsthm}

\DeclareMathOperator{\anz}{anz}

\begin{document}

\section{Thread-Beispiele}

\subsection{Erzeugen von Threads in Java}\label{javathreads}
Schreiben Sie ein Programm in Java, welches acht Threads erstellt und startet. Jeder Thread soll dabei "`Hello World from Thread i"' mit i als Index des Threads im Array ausgeben. Beobachten und interpretieren Sie das Ergebnis.\\
\\
Lösung: \url{https://gist.github.com/lukaswerner/c1a53b8b03fe4d93e1e9dfcfc50c9cb5}

\subsection{Erzeugen von Threads in Java (Ohne Synchronisierung von Strings)}
Erweitern Sie das Programm aus \ref{javathreads} so, dass die Ausgabe des Strings "`Hello World from Thread i"' nicht synchronisiert erfolgt. Schreiben Sie dafür eine Methode, welche einen String Zeichen für Zeichen ausgibt, rufen Sie diese Methode anstelle von System.out.println() auf.\\
\\
Lösung: \url{https://gist.github.com/lukaswerner/61a25246aca3c9fc3fa7c57d86f42af5}

\subsection{Race Condition Beispiel}\label{racecondition}
Schreiben Sie ein Programm in Java, welches $ \geq 10000 $ Threads startet. Jeder Thread soll dabei eine statische Variable (z.B. \emph{z}) hochzählen mit dem \emph{++}-Operator. Geben Sie am Ende diese statische Variable aus und beobachten, sowie interpretieren das Ergebnis.\\
\\
Lösung: \url{https://gist.github.com/lukaswerner/32e344a4c1febc69724d08a60c56064c}

\subsection{Race Condition Lösung}
Erweitern Sie das Programm aus \ref{racecondition} so, dass die Race Condition nicht stattfindet. Implementieren Sie hierfür eine Methode, die gesperrt wird, sobald ein Thread sie ausführt (Stichwort: \emph{synchronized}). Diese Methode soll dann \emph{z++} ausführen.\\
\\
Lösung: \url{https://gist.github.com/lukaswerner/12825318f45e55ec2f3f502a0a49584a}

\section{Amdahls Gesetz}

\paragraph{a)} Finden Sie heraus, welcher Anteil der Zähler-Aufgabe mit $ k = 10000 $ Threads parallelisierbar ist.

\subsubsection*{Lösung}
$ a := $ Zeit bei sequentiellem Ablauf\\
$ b := $ Zeit bei parallelem Ablauf
\begin{align*}
&{} & b & = a \left(1 - p + \frac{p}{n}\right)\\
&\Leftrightarrow & \frac{b}{a} & = 1 - p + \frac{p}{n}\\
&\Leftrightarrow & \frac{b}{a} - 1 & = \frac{p}{n} - p = p \left(\frac{1}{n} -1\right)\\
&\Leftrightarrow & \frac{\frac{b}{a} - 1}{\frac{1}{n} -1} & = p\\
&\Leftrightarrow & p & = \frac{n \left(\frac{b}{a} - 1\right)}{n \left(\frac{1}{n} - 1\right)} = \frac{n \left(\frac{b}{a} - 1\right)}{1 - n} = \frac{n a \left(\frac{b}{a} - 1\right)}{a \left(1 - n\right)} = \frac{n \left(b - a\right)}{a \left(1 - n\right)}
\end{align*}
\noindent
Testwerte:\\
$ a = 0,0000588089s $\\
$ b = 0,6387376938s $\\
\begin{align*}
p = \frac{n \left(b - a\right)}{a \left(1 - n\right)} = \frac{10000 \left(0,6387376938s - 0,0000588089s\right)}{0,0000588089s \left(1 - 10000\right)} = -10861,3281... \approx -1086132,8 \%
\end{align*}

\paragraph{Mit Parallelisierung erheblich langsamer!}

\paragraph{b)} Welchen Anteil erwarten Sie für $ k = 20000 $ Threads?

Zurück gestellt...

\paragraph{c)} Wie weit kann man die Bearbeitung durch Parallelisierung beschleunigen, wenn man beliebig viele Prozessoren zur Verfügung hat?

\begin{align*}
\lim\limits_{n \to \infty} \frac{1}{1 - p + \frac{p}{n}} = \frac{1}{1 - p} = \frac{1}{1 - \frac{n \left(b - a\right)}{a \left(1 - n\right)}} = \ ?
\end{align*}

\section{Verschränkung (1)}

Thread $ p $ habe $ m $ Schritte, Thread $ q $ habe $ n $ Schritte auszuführen.

\paragraph{a)} Geben Sie eine rekursive Definition für die Anzahl $ \anz(m, n) $ der möglichen verschränkten Abläufe von $ p $ und $ q $ an.

% TODO: Evtl. noch Visualisierung

\subsubsection*{Lösung}
\begin{equation*}
\anz(m, n) = 
\begin{cases}
1 & m = 0 \vee n = 0\\
\anz(m - 1, n) + anz(m, n - 1) & \text{sonst}
\end{cases}
\end{equation*}

\paragraph{b)} Finden Sie einen geschlossenen Ausdruck für $ \anz(m, n) $.

\subsubsection*{Lösung}
Wegbeschreibung ist ein Bitvektor mit $ m $ Nullen und $ n $ Einsen, wobei hier gilt:
\begin{center}
$ 0 \ \widehat{=} $ Schritt nach rechts\\
$ 1 \ \widehat{=} $ Schritt nach unten
\end{center}
Die Länge des Bitvektors ist $ n + m $. Isomorph zum Bitvektor mit $ m $ Nullen und $ n $ Einsen sind $ n $ Teilmengen einer $ n + m $ Menge.\\
\\
Beispiel:\\
Bitvektor: $ 011010 $, % TODO: Markieren in Grafik
$ (m + n) $-Menge sei $ \{1, ..., m + n\} $,
dargestellte Teilmente ist $ \{2, 3, 5\} $

\begin{center}
Satz: Die Anzahl der $ k $ Teilmengen einer $ n $-Menge ist $ \binom{n}{k} $.
\end{center} 

Behauptung: $ \anz(m, n) = \binom{m + n}{n} $

\begin{proof}
\begin{align*}
& \text{IA} &  & \text{Sei } n = 0 \Rightarrow \anz(m, 0) = 1 = \binom{m}{0}\\
& \text{IV} &  & \text{Für ein festes } k = m + n \text{ gilt: } \anz(m, n) = \binom{m + n}{n}.\\
& \text{IS} &  & \text{Annahme: } m + n = k + 1\\
& {} &  & \anz(m, n) = \anz(m - 1, n) + \anz(m, n - 1) \overset{\text{IV}}{=} \binom{m - 1 + n}{n} + \binom{m + n - 1}{n - 1}\\
& {} &  & = \binom{k}{n} + \binom{k}{n - 1} \overset{\text{def.}}{=} \binom{k + 1}{n}
\end{align*}
\end{proof}

\paragraph{c)} Schätzen Sie die Größenordnung von $ \anz(m, n) $.

Beim nächsten mal!

\end{document}