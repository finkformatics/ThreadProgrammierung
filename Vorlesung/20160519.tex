% TODO: Rein mergen!
Formel für gegenseitigen Ausschluss kompakter:

$ \pi_{Bel \cup Fr} (PRE(x)) \subseteq PRE((Bel Fr)^*) $

Zugelassen ist z.B. der Ablauf

$ bel_1 fr_2 ant_1 $

Ausgeschlossen ist z.B. $ bel_1 bel_1 fr_2 $

Falls $ y \leq_{pre} x $ und $ x \in X $, dann $ y \in PRE(X) $.

$ X \subseteq A^* $ ist \emph{abgeschlossen unter Präfixen}, falls $ PRE(X) \subseteq X $.

Es gilt: $ PRE(X) $ ist abgeschlossen unter Präfixen.

Gegenseitiger Ausschluss mit Formeln der Prädikatenlogik:

\begin{equation*}
\forall y \in PRE(x) \cap A^* : 0 \leq \#_{Bel}y - \#_{Fr}y \leq 1
\end{equation*}

\begin{center}
\begin{tabular}{c | c}
$ \#_{Bel}y - \#_{Fr}y $ & Bedeutung \\
0 & frei \\
1 & belegt
\end{tabular}
\end{center}

\begin{equation*}
\forall i \in \mathbb{N} : Bel_x^i \leq Fr_x^i \leq Bel_x^{i + 1}
\end{equation*}
\begin{equation*} % TODO: Hier soll eigentlich beispielsweise k über x stehen
\forall k, l \in \mathbb{N}. fr_{i\ x}^k < bel_{j\ x}^l \vee fr_{j\ x}^l < bel_{i\ x}^k
\end{equation*}

Mit Zustandsautomaten:

% TODO: 0 ist Anfangszustand, es gibt auch 1 als Zustand. 0 mappt auf 1 mit Bel, 1 mappt auf 0 mit Fr

\subsubsection*{Lineare temporale Logik}
Hier: Lineare temporale Aussagenlogik.

Syntax: Formeln sind aufgebaut mit:

\begin{itemize}
	\item true, false
	\item Variablen
	\item Verknüpfungen $ \wedge, \vee, \neg $ (weitere Verknüpfungen können damit definiert werden, z.B. $ \Rightarrow, \Leftrightarrow, \oplus $. Endliche Quantoren $ \wedge_{i \in M} $ "`für alle $ i \in M $"', $ \vee_{i \in M} $ "`es existiert $ i \in M $"' mit M endlich.)
	\item temporale Operatoren
	\begin{itemize}
		\item $ \bigcirc $ "`next"' "`im nächten Zustand gilt"'
		\item $ \square $ "`always"' "`in allen zukünftigen Zuständen gilt"'
		\item $ \diamond $ "`eventually"' "`in mind. einem zukünftigen Zustand gilt"' 
	\end{itemize}
\end{itemize}

Beispiel: "`Wer A sagt, muss auch B sagen"'

\begin{equation*}
\square (A \Rightarrow B)
\end{equation*}

"`Never change a running system"'

\begin{equation*}
\square (R \Rightarrow \square R)
\end{equation*}

gleichwertig: $ \square (R \Rightarrow \bigcirc R) $

Semantik:\\
\\
$ \sigma $ sei ein serieller Ablauf\\
$ j $ sei eine natürliche Zahl\\
$ p $ sei eine temporal logische Formel\\
\\ % \models heißt erfüllt
$ (\sigma, j) \models p $ "`Formel $ p $ gilt an Position $ j $ des Ablaufs $ \sigma $"'

Das wird rekursiv definiert durch:

\begin{align*}
& (\sigma, j) \models p \wedge q : \Leftarrow\\
& (\sigma, j) \models p \wedge (\sigma, j) \models q\\
& \text{usw.} \\ % TODO: Zusätzliches Ausrichten nach :<=>
& (\sigma, j) \models \bigcirc p : \Leftrightarrow (\sigma, j + 1) \models p\\
& (\sigma, j) \models \square p : \Leftrightarrow \forall k \geq j: (\sigma, k) \models p\\
& (\sigma, j) \models \diamond p : \Leftrightarrow \exists k \geq j: (\sigma, k) \models p
\end{align*}

Gegenseitiger Ausschluss mit temporal-logischen Formeln:\\
Beispiel:\\
\begin{equation*}
z_x :\Leftrightarrow 0 \leq \#_{Bel}x - \#_{Fr} x \leq 1
\end{equation*}
$ \square z $\\
\\
Kein Verhungern:\\
$ beant_{i\ x} :\Leftrightarrow \#_{ant_i}x > \#_{bel_i}x $\\
"`Thread i hat die Sperre beantragt, aber noch nicht belegt"'\\
\\
$ a_x $ soll bedeuten: Aktion $ a $ ist im Zustand $ x $ soeben ausgeführt worden.\\
\\ % Grafik: 20160519_VerhungernZeitstrahl
Semantik dazu:\\
\begin{equation*}
(\sigma, j) \models a : \Leftrightarrow \sigma(j) = a
\end{equation*}
$ \square (beant_i \Rightarrow \diamond bel_i) $